
\documentclass[preprint,12pt]{elsarticle}

%% Language and font encodings
\usepackage[english]{babel}
\usepackage[utf8x]{inputenc}
\usepackage[T1]{fontenc}
\usepackage{amssymb}



%% Sets page size and margins
\usepackage[a4paper,top=3cm,bottom=2cm,left=3cm,right=3cm,marginparwidth=1.75cm]{geometry}

%% Useful packages
\usepackage{amsmath}
\usepackage{graphicx}
\usepackage[colorinlistoftodos]{todonotes}
\usepackage[colorlinks=true, allcolors=blue]{hyperref}
\usepackage{lineno}

\usepackage{hyperref}
\def\bibsection{\section*{References}}
\linenumbers
%AT:
%AT\newcommand{\main}{.}
%AT\def\biblio{}
%AT\providecommand{\biblio}{\nocite{article-minimal}\bibliographystyle{report}\clearpage\bibliography{\main/references}} 

\journal{Nuclear Instruments and Methods A}

%%\journal{Nuclear Physics B}


\begin{document}
\begin{frontmatter}
\title{Stability of the Scribeline in an Irradiated Silicon Sensor}
\author{Marco Bomben}
\author{Gian-Franco Dalla Betta}
\author{Gabriele Giacomini}

\author{Th\'eophile Boinnard}
%%\author{Wei Chen magari si ringrazia soltanto, come Abdul \corref{cor4}}
\address{Brookhaven National Laboratory, Upton 11973, NY, U.S.A.}


\begin{abstract}
Scribelines are    
\end{abstract}

\begin{keyword}
silicon sensors  \sep electrical characterization \sep high voltage \sep high-energy physics
\end{keyword}

\end{frontmatter}


%%%%%%%%%%%%%%%%%%%%%%%%%%%%%%%%%%%%%%%%%%%%%%%%%

\section{Introduction}

It is well known that current from scribeline is negligible in an irradiated silicon sensor ~\cite{Sadrozinski:2013nja}. 


 Figure~\ref{fig:sketch}. 



\begin{figure}[h!]
\centering
\includegraphics[width=0.5\textwidth]{sketch.jpg}
\caption{\label{fig:sketch} Sketch of the structure}
\end{figure}

%%%%%%%%%%%%%%%%%%%%%%%%%%%%%%%%%%%%%%%%%%%%%%%%%
%%%%%%%%%%%%%%%%%%%%%%%%%%%%%%%%%%%%%%%%%%%%%%%%%

\section{Proposed naming conventions}
\label{sec:naming}
\subsection{Structures}
There are two structures to be used:
\begin{itemize}
\item {\bf U}ndiced (without n$^+$ implant at edge)
\item{\bf T}rench (with n$^+$ implant at edge)
\end{itemize}

\subsection{Damage}
There are two types of damage to be simulated:
\begin{itemize}
    \item {\bf E}dge. For edge the damage can be in two mutually exclusive states: {\bf Y}es (damage is present) and {\bf N}o (no damage is present)
    \item{\bf B}ulk. For bulk damage the state is indicated by the fluence $\Phi$
\end{itemize}

\subsection{Electrodes}
There are three electrodes:
\begin{itemize}
    \item {\bf C}athode
    \item {\bf G}ate (or {\bf G}uard {\bf R}ing)
    \item {\bf S}ubstrate
\end{itemize}

Given the above convention if we want - see below - to study the change in cathode 
current when the bulk is irradiated at fluence $\Phi$ and the edge damage is present 
or not, we will indicate the quantity as:
\begin{equation}
\Delta I_{C}^{(\Phi,Y)} = I_{C,U}^{(\Phi,Y)}-I_{C,T}^{(\Phi,N)}
\end{equation}
indicating that the current when the edge is undamaged is evaluated using the undiced 
structure.


\section{modelization of the edge damage and of the bulk damage}

Figure 1:
Here we can show a 2D map of the electrostatic potential with:
\begin{itemize}
\item Neumann condition: nothing on the edge (device as in an undiced wafer). This is realized using the Undiced structure with no edge damage. Fluence = 0, HV~200V.

\item trench n+ . This is 
realized using the Trench structure. Fluence = 0, HV~200V.

\item  edge with damage. This is realized using the Undiced structure. Fluence = 0, HV~200V.


\end{itemize}
\begin{verbatim}
    --------------------------------------------------------
\end{verbatim}

Figure 2:

possibly also a vertical cutline of the potential along the edge X=699, Y from 0 to 300.
\begin{verbatim}
    --------------------------------------------------------
\end{verbatim}






\section{TCAD simulations}
This is the list of the simulations that we need:


\begin{verbatim}
    --------------------------------------------------------
\end{verbatim}


Figure 3


Let's begin with I-Vs of the scribeline-induced currents at the cathode and at the gate.
We need to subtract the bulk current (which is, for the cathode,  $I_{C,T}^{(\Phi,N)}$).
We can do this subtraction because, as seen in Figure 1, the potential is the same in the two cases of "Trenched with n+" and "damaged scribeline", and so are the depletion regions.

Let's plot the GR currents for several $\Phi$, starting from 0, to 1e15 (whatever, just at least one point below type inversion and one point above type inversion):
\begin{equation}
    \Delta I_{G}^{(\Phi)}=I_{G,U}^{(\Phi,Y)}-I_{G,T}^{(\Phi,N)}
    \end{equation}


let's plot the Cathode currents for several $\Phi$, starting from 0, to 1e15 (whatever, just at least one point below type inversion and one point above type inversion):
\begin{equation}
    \Delta I_{C}^{(\Phi)}=I_{C,U}^{(\Phi,Y)}-I_{C,T}^{(\Phi,N)}
    \end{equation}

\begin{verbatim}
    --------------------------------------------------------
\end{verbatim}





%%%%%%%%%%%%%%%%%
%\begin{itemize}


%\item not irradiated bulk, damaged edge. We see that the current of the GR increases when its depletion region touches the edge. We indicate the currents as (see Section~\ref{sec:naming}):
%\begin{itemize}
%\item $I_{C,U}^{(0,Y)}$ for the cathode,
%    \item $I_{G,U}^{(0,Y)}$ for the gate
%\end{itemize}





%\item we can do a simulation with the trench (passivated edge, or n+ edge) and subtract the gate current to the previous simulation, as to obtain the leakage from the edge only.
%We indicate the currents as:
%\begin{itemize}
%    \item $I_{C,T}^{(\Phi,N)}$ for the cathode,
%    \item $I_{G,T}^{(\Phi,N)}$ for the gate
%\end{itemize}


%\item  irradiated bulk (maybe a few fluences?), damaged edge. We see that the  current of the gate (due only to the edge) is lower than in the case of non-irradiated bulk.
%We indicate the currents as:
%\begin{itemize}
 %   \item $I_{C,U}^{(\Phi,Y)}$
%    \item $I_{G,U}^{(\Phi,Y)}$
%\end{itemize}
%We want to study the following difference:
%\begin{equation}
%    \Delta I_{G}^{(\Phi)}=I_{G,U}^{(\Phi,Y)}-I_{G,T}^{(\Phi,N)}
%    \end{equation}
%    \begin{equation}
%    \Delta I_{C}^{(\Phi)}=I_{C,U}^{(\Phi,Y)}-I_{C,T}^{(\Phi,N)}
%\end{equation}

%\item also here  we may want to simulate the situation with the edge passivated (n$^+$ edge).
We want to study the following difference:
%\begin{itemize}
%    \item $I_{C,T}^{(\Phi,Y)}$
%    \item $I_{C,T}^{(\Phi,Y)}$
%\end{itemize}

%\end{itemize}
%%%%%%%%%%%%%%%%%%%%%
2D maps  from Tonyplot:
in irradiated vs non irradiated bulk (both with damaged edge):

Figure 4:

Hole current in the geometry of damaged scribeline only.
- $\Phi =0 $ and V= enough to reach the scribeline (and let's use just this Voltage from now on)
- $\Phi$ after type inversion for V at the peak of the Cathode current
- $\Phi$ after type inversion for V where cathode current has plateau, possibly the same voltage as in the case $\Phi =0 $ above.

\begin{verbatim}
    --------------------------------------------------------
\end{verbatim}


Figure 5:
Electrostatic Potential 
- $\Phi =0 $ and V= enough to reach the scribeline (same voltage used above)
- $\Phi$ after type inversion for same V as above


\begin{verbatim}
    --------------------------------------------------------
\end{verbatim}
Figure 6:
Hole concentration 
- $\Phi =0 $ and V= enough to reach the scribeline (same voltage used above)
- $\Phi$ after type inversion for same V as above

\begin{verbatim}
    --------------------------------------------------------
\end{verbatim}

Figure 7:

Ionized traps density for $\Phi$ after type inversion, V same as above

\begin{verbatim}
    --------------------------------------------------------
\end{verbatim}

Figure 8:


charge density
- $\Phi =0 $ and V= enough to reach the scribeline, same as above
- $\Phi$ after type inversion for same V as above

\begin{verbatim}
    --------------------------------------------------------
\end{verbatim}






%%%%%%%%%%%%%%%%%%%%%%%%%%%%%%%%%%%%%%%%%%%%%%%%%
%%%%%%%%%%%%%%%%%%%%%%%%%%%%%%%%%%%%%%%%%%%%%%%%%
\section{Electrical Characterization}

 
 %%%%%%%%%%%%%%%%%
% \begin{table}
%\begin{tabular}{|c|c|c|c|}
%\hline
%A  & B  & C & D    \\
% \hline
% 2.50  & 2.3 & 0.92 & 450   \\
% 2.75 & 2.5 & 0.91 & 380   \\
% 3.00    & 2.7 &  0.90 & 300 \\ 
% 3.25 & 2.9 &  0.89 & 200   \\
% 3.50  & 3.2 &  0.91 & 50  \\
% \hline
%\end{tabular}
% \caption{this is a template of a table, if we need it}
%  \label{table:table}
%\end{table}


%%%%%%%%%%%%%%%%%%%%%%%%%%%%%%%%%%%%%%%%%%%%%
%%%%%%%%%%%%%%%%%%%%%%%%%%%%%%%%%%%%%%%%%%%%%
\section{ Measurements}





\section{Conclusions}



\section{Acknowledgements}
This material is based upon work supported by the U.S. Department of Energy under grant DE-SC0012704.



\bibliographystyle{elsarticle-num}
\bibliography{biblio}{}
%AT:
%\bibliographystyle{report}
%\bibliography{\main/references}

\end{document}
